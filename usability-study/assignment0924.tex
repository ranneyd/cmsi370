\documentclass[a4paper,12pt]{article}
\usepackage{amssymb}
\usepackage{geometry}
\usepackage{graphicx}
\usepackage{colortbl}
\usepackage{wrapfig}

\geometry{margin=1in}



\begin{document}

Dustin Kane

CMSI 370-01

September 24, 2015

\begin{center}
\section*{Assignment 0924: Usability Study}
\subsection*{Comparing Google Drive and Microsoft OneDrive}
\end{center}

\section{Introduction}

Cloud services are ubiquitous these days. People now have much of the software they use everyday available to them wherever they have internet access. A key component of this cloud-based lifestyle is cloud file hosting. Files are an important part of internet use, so it's only natural that files play an important role in the cloud-computing shift. There are a variety of file-hosting cloud services, and many offer similar features. But two cloud file-hosting services are notable because they provide robust file editing utilities. These services are Google Drive by Google and OneDrive by Microsoft. 

Google Drive began as Google's competitor to Dropbox (perhaps the most well-known cloud-based file-hosting service), offering very similar features. But Google soon married Drive with Google Docs, software that is synonymous with online file editing, under the Google Drive branding.

Microsoft's OneDrive platform was also created as an alternative to Dropbox, one that supposedly integrates well with the Windows ecosystem. But one of OneDrive's key strengths is its integration with Microsoft Office. OneDrive has in-browser versions of Microsoft Word, Excel, and Powerpoint, each of which has a desktop client that is the undisputed leader in its area. These in-browser versions are almost identical to their desktop counterparts and support a surprisingly robust feature set.

Google Docs is the leader in online word processing, whereas Microsoft Word is the leader in offline word processing (and has been for decades). Both are integrated into an online file-hosting platform. Our goal was to see which software provided an interface that was most effective at creating, sharing, and collaborating on documents, as well as providing support for version control.

\section{Usability Metrics}

\subsection{Our Focus}
Our goal was to find which online file-hosting and editing software was the most usable. We chose to focus on four key features that represent a wide spectrum of use cases: creating and editing a document, sharing the document with another person, collaborating on that document with another person using comments, and reverting to previous version of the file. The first was a natural choice: no feature of the editing software is usable unless the user can create and edit files. The second hits file sharing: a key benefit of using cloud services. The third expands on the sharing concept to look at tools used for collaborating on a single document between multiple people. The final feature is a lesser-known and more complex feature that is incredibly powerful if used properly.

\subsection{Our Method}

We developed three tasks to be performed in both clients by our test subjects. First, the user had to create a new document, name it ``cats'', type ``dog'' for the body, and then share the document with the email address ``kcgotfre@gmail.com''. For the second task, the user had to highlight the word ``dog'' from the first task and leave a comment on it that said ``woof woof''. We explained to the users first what a comment was and how it was meant to leave notes to collaborators without actually editing the document. For the last task, we had created a separate document that we had edited multiple times and we asked the user to revert it to any previous version. For each of these tasks we recorded from the time they began to the time they hit the last necessary button to complete the task. We also counted the number of errors each user made in each task. We had them do all three tasks for a service together, then switch to the other. The first service we tested for each user was chosen randomly at the start of the test.

We noted which users had used the software before and which had not to better assess learnability vs efficiency. We retested many subjects who had no experience with the software again, this time to test efficiency rather than learnability.

\subsection{Our Metrics}

 We focused on three metrics, $learnability$, $efficiency$, and $errors$. The first, learnability, measures how easy an interface is to figure out. A highly learnable interface should allow a brand new user to easily figure out how to do what he or she wants to do. The second, efficiency, measures how fast someone who is knowledgeable of the software can do tasks they know exactly how to do. If someone knows exactly what to do, the only limiting factors involve the limitations of the user interface. The final metric is errors, which measures how often a user does something with an expectation that isn't met. An interface should make sense and be logical to the user. It should correspond to their mental model of what the system should do, and if it doesn't, it isn't an effective interface.

\subsection{Our Data}

A few notes about the following table. First, certain test subjects are listed twice. This is because some subjects who were new to the software were tested again to test efficiency. A Y in ``XP'' means the user was experienced in that software, an N means they were not. ``E'' is the number of errors that occurred during the test whose time is to the left. ``Create'' represents the time the first task took: creating, editing, and sharing a document. ``Comnt'' is the time the second task took: leaving a comment for a collaborator. ``Revert'' is the time the third task took: reverting the document to an older version.

\begin{table}[h]
\footnotesize
\centering
\begin{tabular}{|c|c|c|c|c|c|c|c|c|c|c|c|c|c|c|}
		\hline
		Tst& \multicolumn{7}{c}{Microsoft OneDrive} & \multicolumn{7}{|c|}{Google Drive} \\
		Sub & XP&Create&E&Comnt&E&Revert&E&XP&Create&E&Comnt&E&Revert&E\\ \hline
Ka1&N&00:35.1&0&00:10.2&0&01:15.1&2&Y&00:13.5&0&00:10.4&0&00:12.0&0\\ \hline
Ka2&N&00:45.1&0&00:51.2&1&00:45.6&0&N&00:30.6&0&00:13.8&0&00:14.9&0\\ \hline
Ka3&N&00:23.2&0&00:26.5&0&00:41.3&0&N&00:19.7&0&00:52.6&1&00:47.2&0\\ \hline
Ka3&Y&00:13.4&0&00:17.8&0&00:24.4&0&Y&00:11.6&0&00:08.7&0&00:19.5&0\\ \hline
Su1&N&01:02.7&0&00:33.0&1&00:47.6&0&Y&00:20.4&0&00:14:5&2&00:24.7&0\\ \hline
Su1&Y&00:20.6&0&00:07.9&0&00:56.5&0&Y&\cellcolor[gray]{0.3}&\cellcolor[gray]{0.3}&\cellcolor[gray]{0.3}&\cellcolor[gray]{0.3}&\cellcolor[gray]{0.3}&\cellcolor[gray]{0.3}\\ \hline
Su2&N&00:59.0&0&00:24.7&0&00:24.8&0&Y&00:32.0&0&00:17.0&0&00:19.1&0\\ \hline
Su2&Y&00:19.3&0&00:14.7&0&00:45.3&0&N&\cellcolor[gray]{0.3}&\cellcolor[gray]{0.3}&\cellcolor[gray]{0.3}&\cellcolor[gray]{0.3}&\cellcolor[gray]{0.3}&\cellcolor[gray]{0.3}\\ \hline
Su3&N&00:34.4&0&00:11.4&0&00:36.6&0&Y&00:16.2&0&00:10.7&0&00:54.4&0\\ \hline
Su4&N&00:44.8&0&00:13.5&0&01:29.2&0&Y&00:17.9&0&00:07.5&0&00:15.7&0\\ \hline
Su4&Y&00:23.5&0&00:07.3&0&00:36.0&0&Y&\cellcolor[gray]{0.3}&\cellcolor[gray]{0.3}&\cellcolor[gray]{0.3}&\cellcolor[gray]{0.3}&\cellcolor[gray]{0.3}&\cellcolor[gray]{0.3}\\ \hline
Su5&N&00:58.8&0&00:10.8&0&00:37.7&0&Y&00:11.3&0&00:08.9&0&00:09.3&0\\ \hline
Su5&Y&00:15.1&0&00:15.9&0&00:27.3&0&Y&\cellcolor[gray]{0.3}&\cellcolor[gray]{0.3}&\cellcolor[gray]{0.3}&\cellcolor[gray]{0.3}&\cellcolor[gray]{0.3}&\cellcolor[gray]{0.3}\\ \hline
Su6&N&02:11.2&0&00:16.3&0&00:48.1&0&N&01:06.3&0&00:09.2&0&00:52.0&0\\ \hline
Su6&Y&00:34.2&0&00:08.0&0&00:28.6&0&Y&00:28.3&0&00:05.8&0&00:12.7&0\\ \hline
Su7&N&00:47.4&0&01:02.0&0&03:22.3&0&Y&00:29.9&0&00:10.6&0&00:30.3&0\\ \hline
Du1&N&03:30.0&1&00:42.0&1&03:47.0&2&N&01:43.0&1&01:33.0&1&00:30.0&0\\ \hline
Du2&N&01:58.0&0&00:16.0&0&01:34.0&2&N&02:22.0&2&00:17.0&0&01:35.0&2\\ \hline
Du3&Y&00:24.4&0&00:11.2&1&00:14.8&0&Y&00:22.1&0&00:07.0&1&00:19.6&1\\ \hline
Ku1&N&00:34:0&1&00:24:3&0&00:46:0&2&N&00:28:2&0&00:15:3&0&00:32:2&0\\ \hline
Ku1&Y&00:15:1&0&00:17:5&0&00:30:4&0&Y&00:11:5&0&00:11:1&0&00:19:3&0\\ \hline
Ku2&N&01:10:2&2&00:32:0&0&00:48:2&1&N&00:46:2&0&00:35:2&0&00:38:3&1\\ \hline
Ku3&N&00:53:4&0&00:22:7&0&00:36:2&0&N&00:38:3&0&00:23:0&0&00:30:4&0\\ \hline
Ku3&Y&00:28:5&0&00:15:2&0&00:30:3&0&Y&00:30:4&0&00:18:2&0&00:22:4&0\\ \hline
Ku4&N&00:45:8&0&00:23:4&0&01:03:1&1&N&00:36:1&0&00:28:7&0&00:40:1&0\\ \hline
Ku5&Y&00:10:2&0&00:15:2&0&00:22:5&1&Y&00:12:3&0&00:05:5&0&00:09:2&0\\ \hline
\end{tabular}
\caption{Test Data}
\end{table}
\subsection{Our Results}

The following table includes the average times and error counts for each task for each service for new users, experienced users, and all users together.

\begin{table}[h]
\small
\centering
\begin{tabular}{|c|c|c|c|c|c|c|c|c|c|c|c|c|}
\hline
& \multicolumn{6}{c}{Microsoft OneDrive} & \multicolumn{6}{|c|}{Google Drive}\\ \hline
&Create&E&Comnt&E&Revert&E&Create&E&Comnt&E&Revert&E\\ \hline
New&00:52.3&4&00:26.3&3&00:57.7&10&00:56.7&3&00:25.0&2&00:35.6&3\\ \hline
Exp.&00:20.4&0&00:13.1&1&00:31.6&1&00:19.8&0&00:10.5&3&00:20.6&1\\ \hline
\end{tabular}
\caption{Test Results and Aggregates}
\end{table}


With a few exceptions, Google Drive got better results across the board. OneDrive was marginally faster for creating new documents, and experienced users made more errors leaving comments in Google Drive. In all other areas Google Drive performed the same as or better than OneDrive. 

It's worth noting that OneDrive was, almost, on par with Google Drive in most areas except in file versioning. In that area (an area where the interfaces vary significantly between the two platforms) OneDrive performed much worse; it even had took an average of over 20 seconds for new users to figure out file reverting. It seems safe to say that Google Drive's version control interface is superior in all three metrics. 

For the other tasks, the results were closer. For learnability (looking at new users), OneDrive performed better than Google Drive in creating, editing and sharing by just over four seconds, but in leaving comments for collaboration, Google Drive won out by just a hair. 

For efficiency (looking at experienced users), OneDrive lost to Google Drive in file creation, editing, and sharing by fractions of a second, whereas it lost by a little under three seconds for commenting. In the averages for both new and experienced users, Google Drive beat OneDrive by about five seconds on average in the first and second tasks.

For errors, OneDrive had one more error in total for the first task, whereas Google Drive had one more error in leaving a comment.

\section{Heuristic Evaluation}

\subsection{The First Task}

The first task involved creating, editing, renaming, and sharing a file. In this area OneDrive beat Google Drive for new users by 4.4 seconds, and lost by just 0.6 seconds experienced users. 

I believe OneDrive's success among new users is due, primarily, to its similarity with Microsoft Word. When we asked users if they had used OneDrive before, we did not ask if they had used Microsoft Word. But the software is so ubiquitous that asking it would have likely been pointless; everyone has used Microsoft Word, and many people have used it extensively. To a new user, OneDrive's interface, which is highly reminiscent of Microsoft Word, should feel very familiar (see figures 1 and 2).

\begin{figure}[h]
\centering
\includegraphics[width=\textwidth]{onedrive}
\caption{OneDrive's ``Word Online'' document editor. Note it is very similar in appearance to Microsoft Word 2013}
\end{figure}

As to Google Drive's success among experienced users, I suspect this is due to Google Drive being slightly more responsive than OneDrive. The first task is incredibly basic and anyone who claims to be experienced in either of these services should know how to do these very easily. At an average difference of 0.6 seconds in Google Drive's favor, I do not think Google Drive has an interface that made this task particularly more efficient. In my experience, however, loading the ``share'' dialog takes a little bit longer in OneDrive, which could explain the difference.

\begin{figure}[h]
\centering
\includegraphics[width=\textwidth]{word}
\caption{Microsoft Word 2013. Note the similarities between this and Figure 1}
\end{figure}

\subsection{The Second Task}

\begin{wrapfigure}{l}{0.5\textwidth}
\centering
\includegraphics[width=0.5\textwidth]{drivecomment}
\caption{Insert Menu in Google Drive.}
\end{wrapfigure}

The first task required the user to highlight the word ``dog'' and leave a comment for a collaborator.

For new users, Google Drive was 1.3 seconds faster. In both services, inexperienced users look for the insert menu. In Google Drive this is a traditional drop down menu, and in OneDrive's Word Online it's a tab. Both of these areas contain the comment button. Google Drive uses a vertical list view. In this view it's much easier to quickly look down the list until you find ``Comment'' (see figure 3). But Word Online's view is cluttered with symbols that make it hard to find the Comment button at first glance; it takes some searching of the page to find it (see figure 4).  I suspect many users pause after they hit the insert tab in OneDrive while they try to find the button, whereas Google Drive requires only a quick glance.


\begin{figure}[h]
\centering
\includegraphics[width=\textwidth]{onedrivecomment}
\caption{Insert tab of OneDrive's Word Online. Notice how hard it is to find the ``Comment'' button at first glance.}
\end{figure}


Experienced users were, on average, 2.6 seconds faster at this task in Google Drive than in OneDrive. The reasoning for this, I think, is very clear. Google Drive has a dedicated button in its main toolbar for leaving a comment; it does not require entering the insert menu (see figure 5).

\begin{figure}[h]
\centering
\includegraphics[width=\textwidth]{drivecommentbutton}
\caption{The main toolbar in Google Drive with the Comment button circled in red.}
\end{figure}

\subsection{The Third Task}

The third task was a disaster for OneDrive. At 22.1 seconds slower for new users and 11 seconds slower for experienced users with an alarming seven more errors in total, Google Drive is undoubtedly superior. I believe this is due to a variety of factors.

From within the editor, finding version history in OneDrive is nearly impossible. The user must hit ``File'' and then ``Info'', and under info is the option to look at previous versions. Many users ignore the Info option in file, some giving up before they even look at it.

While Google Drive has ``See revision history'' under its file menu, experienced users actually know that where Google Drive shows you how long ago the file was edited, this is actually a button that takes you to the version control view (see figure 6). Using this button makes the process incredibly quick.

\begin{figure}[h]
\centering
\includegraphics[width=\textwidth]{lastedited}
\caption{Experienced Google Drive users know that the yellow-highlighted text is actually a button that leads to the revision history view.}
\end{figure}


\begin{wrapfigure}{l}{0.5\textwidth}
\centering
\includegraphics[width=0.4\textwidth]{showdetail}
\caption{Google Drive only shows $some$ revisions. To show all of them, the user must hit the button at the bottom that says ``Show more detailed revisions''.}
\end{wrapfigure}

I would like to point out that this test was a little unfair to OneDrive for two reasons. First, we required users to revert the file from within the editor. When using OneDrive's folder view, outside the editor, I believe reverting a file is much easier in OneDrive because you can simply right click on the file and hit ``Version History''. Google Drive's right-click menu does not have this option. Instead, it has an ``activity'' list on the right-hand side when you select a file. But it is to the right and out of the way, it is not clear what the activity is or what it means, and it isn't clear that the previous versions are clickable, viewable, and revertable.

Second, to simplify the test, we told the user they could choose any previous version. However, Google Drive does not show all revisions by default. To see a list of all revisions, the user must hit a button at the bottom of the list that says ``Show more detailed revisions''. Not only is this text misleading (it could mean show a larger quantity of revisions or show the currently displayed revisions in more detail), but it is unclear to the user that the list before hitting this button isn't the complete list. I know I, personally, spent minutes when setting up the tests trying to figure out why it wasn't showing all the revisions, only to discover the function of the button at the bottom. In OneDrive, one the user (miraculously) finds the version control view, it lists all the versions.

\section{Conclusion}

The numbers certainly suggest Google Drive has the superior interface. Though in most areas it was only marginally better, Google Drive was fairly consistently faster among experienced and inexperienced users, suggesting better learnability and efficiency. It is clear that OneDrive has serious problems in the way it handles version control, but the nature of our experiment was inherently slightly biased toward Google Drive.

I'm not convinced, however, that this means Google Drive is the superior service. Google Drive has a simpler interface than OneDrive, but that's partially because it has far fewer features. Google Drive is often criticized by Microsoft Word aficionados for being underpowered. Microsoft's Word Online, though a slimmed down version of the desktop software, has a surprisingly robust feature set. Though OneDrive's interface may be a little less forgiving when working with a document that only contains the word ``dog'', its based on the interface of what is universally agreed to be the world's best desktop word processor. For more complex tasks such as working with image positioning or making complex tables, I doubt that Google Drive is as learnable or efficient as OneDrive. But for the features we tested, the numbers don't lie.

\end{document}