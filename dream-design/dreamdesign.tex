\documentclass[a4paper,12pt]{article}
\usepackage{amssymb}
\usepackage{geometry}
\usepackage{graphicx}
\usepackage{enumerate}
\usepackage{amsmath}
\geometry{margin=1in}



\begin{document}

Dustin Kane

CMSI 370-01

November 24, 2015

\begin{center}
\section*{Assignment 1124: Dream Design}
\subsection*{The Ultimate, Interactive Heads-Up Display}
\end{center}

\section{Introduction}
\subsection{A Bicycle For Our Minds}

Steve Jobs, the cofounder of Apple Computers, liked to tell a story. He found a study that measured the distance versus energy consumption for various animals and found the condor to be the most efficient. 

\section{Sources}

``The holograms you’ll see with Microsoft HoloLens can appear life-like, and can move, be shaped, and change according to interaction with you or the physical environment in which they are visible. Use gestures to create, shape, and size holograms. Use your gaze to navigate and explore. Use your voice to communicate with your apps. Microsoft HoloLens understands your movements, gaze, and voice, enabling you to interact with content and information naturally. Using holograms, you can place your digital content, such as apps, information, and even multi-dimensional videos, in the physical space around you, so you can interact with it.''

``The HPU is custom silicon that processes a large amount of data per second from the sensors. Microsoft HoloLens understands gestures and where you look, and maps the world around you, all in real time.''
\begin{enumerate}
    \item https://www.brainpickings.org/2011/12/21/steve-jobs-bicycle-for-the-mind-1990/
	\item https://www.microsoft.com/microsoft-hololens/en-us/faq
	\item https://www.microsoft.com/microsoft-hololens/en-us/hardware
	\item https://en.wikipedia.org/wiki/Head-up\_display
\end{enumerate}


\end{document}